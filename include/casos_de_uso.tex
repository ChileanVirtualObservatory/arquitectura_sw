\section{Casos de Uso ChiVO}
\noindent En la captura de requerimientos y casos de uso participaron los siguientes
astrónomos:
\begin{itemize}
	\item Diego Mardones, Universidad de Chile.
	\item Lars Nyman, Atacama Large Milimiter/submilimiter Array.
	\item Neil Nagar, Universidad de Concepción.
	\item Nelson Padilla, Pontificie Universidad Católica. 
	\item Juan de Santander, Universidad de Granda, España.
	\item Amelia Bayo, European Souther Observatory.
\end{itemize}

\noindent El informe de requerimientos (Hito Anterior) se puede encontrar en
\cite{hrequerimientos}.

\noindent Para abordar de mejor forma los casos de uso, se mencionarán algunos
conceptos relevantes:
\begin{itemize}
	\item Los datos astronómicos en sus diferentes formatos se dividen en:
Metadata y Binarios. La metadata son datos que describen los datos de la
observación, y los binarios son los datos obtenidos de la observación.
	\item En la primera etapa el proyecto trabajará con archivos FITS.
	\item Las búsquedas en el sistema se realizan sobre la metadata. El
resultado de una búsqueda es un set de datos candidatos que coinciden con los
parámetros fijados en la búsqueda.
	\item La interacción final con el sistema es cuando el usuario decide qué
datos necesita, los selecciona y los descarga de forma local.
\end{itemize}

\subsection{Casos de Uso Generales}
Los siguientes casos de usos involucran de manera general a varios
requerimientos, por lo que se presentan en una categoría especial. 

\noindent\textbf{Caso de uso \#1}: \\
\textbf{Objetivo}: Filtrar los resultados de la búsqueda en el Portal Web. \\
\textbf{Actor}: Usuario. \\
\textbf{Necesidad}: Esencial. \\
\textbf{Prioridad}: Alta. \\
\textbf{Requerimientos Referenciados}: del 1 al 7. \\
\textbf{Descripción}: Ya que el usuario puede recibir una gran cantidad de
resultados, éste debe ser capaz de realizar un filtro sobre alguna columna de
los datos recibidos al realizar la búsqueda para así poder identificar los
datos que se encuentre buscando. Este filtro puede ser realizado en el mismo
navegador del usuario o a través de una nueva consulta. 
\vspace{1.0cm}

\noindent\textbf{Caso de uso \#2}: \\
\textbf{Objetivo}: Descargar datos desde el Portal Web. \\
\textbf{Actor}: Usuario. \\
\textbf{Necesidad}: Esencial. \\
\textbf{Prioridad}: Alta. \\
\textbf{Requerimientos Referenciados}: del 1 al 7. \\
\textbf{Descripción}: Una vez que el usuario encuentre los datos que requiere, éste procede a descargarlos. Para ello el sistema provee un enlace directo a la fuente de los datos.
\vspace{1.0cm}

\noindent\textbf{Caso de uso \#3}: \\
\textbf{Objetivo}: Visualizar una representación gráfica que compare los metadatos de los resultados de la búsqueda. \\
\textbf{Actor}: Usuario. \\
\textbf{Necesidad}: Esencial. \\
\textbf{Prioridad}:  Media. \\
\textbf{Requerimientos Referenciados}: del 1 al 7. \\
\textbf{Descripción}: Una vez que el usuario reciba un conjunto de resultados, podrá visualizarlos en gráficos acordes al tipo de búsqueda y en base a los metadatos recibidos por la búsqueda.
\vspace{1.0cm}

\noindent\textbf{Caso de uso \#4}: \\
\textbf{Objetivo}: Visualizar un producto o subproducto de los datos de la observación presente en los metadatos. \\
\textbf{Actor}: Usuario. \\
\textbf{Necesidad}: Esencial. \\
\textbf{Prioridad}: Alta. \\
\textbf{Requerimientos Referenciados}: del 1 al 7. \\
\textbf{Descripción}: Visualización de la forma geométrica de las observaciones (rectangulares o redondas); cobertura UV; Calibración de paso de banda;Espectro observado;Imagen observada en un plano del producto cuando sea posible (podría ser parte de análisis en vez de observación);
\vspace{1.0cm}

\noindent\textbf{Caso de uso \#5}: \\
\textbf{Objetivo}: Análisis de cubo FITS. \\
\textbf{Actor}: Usuario.\\
\textbf{Necesidad}: Esencial. \\
\textbf{Prioridad}: Alta. \\
\textbf{Requerimientos Referenciados}: del 1 al 7. \\
\textbf{Descripción}: Selecciona pixeles; proyección en una dimensión;flujo en un área;series de tiempo.
\vspace{1.0cm}

\noindent\textbf{Caso de uso \#6}: \\
\textbf{Objetivo}: Análisis de ASDM. \\
\textbf{Actor}: Usuario. \\
\textbf{Necesidad}: Deseable. \\
\textbf{Prioridad}: Media. \\
\textbf{Requerimientos Referenciados}: del 1 al 7. \\
\textbf{Descripción}: Selecciona voxeles; proyección en una dimensión;flujo en un área;series de tiempo.
\vspace{1.0cm}

\noindent\textbf{Caso de uso \#7}: \\
\textbf{Objetivo}: Los resultados de la búsqueda deben ser analizables para secuencias de tiempo. \\
\textbf{Actor}: Usuario. \\
\textbf{Necesidad}: Esencial. \\
\textbf{Prioridad}: Baja. \\
\textbf{Requerimientos Referenciados}: del 1 al 7. \\
\textbf{Descripción}: El usuario luego de realizar una búsqueda por tipo y/o subtipo de objeto, puede obtener los resultados de la búsqueda para ser analizados como secuencias de tiempo.
\vspace{1.0cm}

\subsection{Buscar por coordenada o región del cielo}
\noindent\textbf{Caso de uso \#8}: \\
\textbf{Objetivo}: Ingresar al Portal Web y realizar una búsqueda por coordenadas. \\
\textbf{Actor}: Usuario. \\
\textbf{Necesidad}: Esencial. \\
\textbf{Prioridad}: Alta. \\
\textbf{Requerimientos Referenciados}: 1. \\
\textbf{Descripción}: El usuario ingresa al Portal Web, rellena los campos de coordenadas y radio angular o región de cielo y realiza una búsqueda. Los parámetros de coordenada pueden pertenecer al sistema ecuatorial (J2000 o B1950), eclíptico, galáctico o supergaláctico.
\vspace{1.0cm}

\noindent\textbf{Caso de uso \#9}: \\
\textbf{Objetivo}: Realizar una búsqueda de listado de coordenadas. \\
\textbf{Actor}: Usuario. \\
\textbf{Necesidad}: Esencial. \\
\textbf{Prioridad}: Alta. \\
\textbf{Requerimientos Referenciados}: 1. \\
\textbf{Descripción}: El usuario ingresa al Portal Web, ingresa una lista de coordenadas y radios angulares o regiones de cielo y realiza una búsqueda. Los parámetros de coordenada pueden pertenecer al sistema ecuatorial (J2000 o B1950), eclíptico, galáctico o supergaláctico. También el usuario puede subir un archivo con un formato establecido por el sitio con el listado de coordenadas.
\vspace{1.0cm}

\subsection{Buscar por nombre o tipo de objeto}
\textbf{Caso de uso \#10}: \\
\noindent\textbf{Objetivo}: Ingresar al Portal Web y realizar una búsqueda por nombre en base a Sesame. \\
\textbf{Actor}: Usuario.\\
\textbf{Necesidad}: Esencial.\\
\textbf{Prioridad}: Alta.\\
\textbf{Requerimientos Referenciados}: 2. \\
\textbf{Descripción}: El usuario ingresa al Portal Web, rellena el campo de nombre según los nombres definidos en Sesame y realiza una búsqueda.
\vspace{1.0cm}

\noindent\textbf{Caso de uso \#11}: \\
\textbf{Objetivo}: Ingresar al Portal Web y realizar una búsqueda por nombre en base a catálogo de un instrumento. \\
\textbf{Actor}: Usuario.\\
\textbf{Necesidad}: Esencial.\\
\textbf{Prioridad}: Alta.\\
\textbf{Requerimientos Referenciados}: 2. \\
\textbf{Descripción}: El usuario ingresa al Portal Web, rellena el campo de nombre según los nombres definidos en Catálogos específicos de un instrumento y realiza una búsqueda.
\vspace{1.0cm}

\noindent\textbf{Caso de uso \#12}: \\
\textbf{Objetivo}: Ingresar al Portal Web y realizar una búsqueda por tipo de objeto en un área del cielo. \\
\textbf{Actor}: Usuario.\\
\textbf{Necesidad}: Esencial.\\
\textbf{Prioridad}: Alta.\\
\textbf{Requerimientos Referenciados}: 2. \\
\textbf{Descripción}: El usuario ingresa al Portal Web, rellena los campos de tipo y/o subtipos de objetos y un área del cielo y realiza una búsqueda.
\vspace{1.0cm}

\subsection{Buscar por metadatos espectrales}
\noindent\textbf{Caso de uso \#13}: \\
\textbf{Objetivo}: Ingresar al Portal Web y realizar una búsqueda espectral extragaláctica. \\
\textbf{Actor}: Usuario.\\
\textbf{Necesidad}: Esencial.\\
\textbf{Prioridad}: Alta.\\
\textbf{Requerimientos Referenciados}: 3. \\
\textbf{Descripción}: El usuario ingresa al Portal Web, rellena los campos de banda o rango de frecuencia;  y/o líneas espectrales y corrimiento al rojo (z); y/o resolución espectral y/o ruido y realiza una búsqueda.
\vspace{1.0cm}

\noindent\textbf{Caso de uso \#14}: \\
\textbf{Objetivo}: Ingresar al Portal Web y realizar una búsqueda espectral galáctica. \\
\textbf{Actor}: Usuario.\\
\textbf{Necesidad}: Esencial.\\
\textbf{Prioridad}: Alta.\\
\textbf{Requerimientos Referenciados}: 3. \\
\textbf{Descripción}: El usuario ingresa al Portal Web, rellena los campos de banda o rango de frecuencia;  y/o líneas espectrales y velocidad radial (v\_r); y/o resolución espectral y/o ruido y realiza una búsqueda.  Línea espectral incluye campos de moléculas, transición de moléculas (vibracional, rotacional o electrónica) o frecuencia en reposo de la línea espectral.
\vspace{1.0cm}

\noindent\textbf{Caso de uso \#15}: \\
\textbf{Objetivo}: Realizar una búsqueda de listado de frecuencias. \\
\textbf{Actor}: Usuario.\\
\textbf{Necesidad}: Esencial.\\
\textbf{Prioridad}: Alta.\\
\textbf{Requerimientos Referenciados}: 3. \\
\textbf{Descripción}: El usuario ingresa al Portal Web, ingresa una lista de frecuencias y realiza una búsqueda. También el usuario puede subir un archivo con un formato establecido por el sitio con el listado de frecuencias.
\vspace{1.0cm}

\subsection{Buscar por metadatos espaciales}
\noindent\textbf{Caso de uso \#16}: \\
\textbf{Objetivo}: Ingresar al Portal Web y realizar una búsqueda por resolución angular y/o campos de visión. \\
\textbf{Actor}: Usuario.\\
\textbf{Necesidad}: Esencial.\\
\textbf{Prioridad}: Media.\\
\textbf{Requerimientos Referenciados}: 4. \\
\textbf{Descripción}: El usuario ingresa al Portal Web, rellena los campos de resolución angular y/o campos de visión y realiza una búsqueda.
\vspace{1.0cm}

\subsection{Buscar por metadatos temporales}
\noindent\textbf{Caso de uso \#17}: \\
\textbf{Objetivo}: Ingresar al Portal Web y realizar una búsqueda relacionada con cuando fue realizada la observación. \\
\textbf{Actor}: Usuario.\\
\textbf{Necesidad}: Deseable.\\
\textbf{Prioridad}: Baja.\\
\textbf{Requerimientos Referenciados}: 5. \\
\textbf{Descripción}: El usuario ingresa al Portal Web, rellena los campos de tiempo, cantidad de observaciones de un objeto y/o el intervalo entre las observaciones y realiza la búsqueda.
\vspace{1.0cm}

\noindent\textbf{Caso de uso \#18}: \\
\textbf{Objetivo}: Ingresar al Portal Web y realizar una búsqueda relacionada con nivel de ruido, duración de la observación o tiempo de integración.\\
\textbf{Actor}: Usuario.\\
\textbf{Necesidad}: Esencial.\\
\textbf{Prioridad}: Alta.\\
\textbf{Requerimientos Referenciados}: 5. \\
\textbf{Descripción}: El usuario ingresa al Portal Web, rellena los campos de nivel de ruido, duración de la observación o tiempo de integración y realiza una búsqueda.
\vspace{1.0cm}

\subsection{Buscar por polarización}
\noindent\textbf{Caso de uso \#19}: \\
\textbf{Objetivo}: Ingresar al Portal Web y realizar una búsqueda por parámetros de Stokes. \\
\textbf{Actor}: Usuario.\\
\textbf{Necesidad}: Esencial.\\
\textbf{Prioridad}: Baja.\\
\textbf{Requerimientos Referenciados}: 6. \\
\textbf{Descripción}: El usuario ingresa al Portal Web, rellena los campos de parámetros de Stokes necesarios (I, Q, U y/o V) o de polarización izquierda o derecha y realiza la búsqueda.
\vspace{1.0cm}

\subsection{Cruzamiento de información}
\noindent\textbf{Caso de uso \#20}: \\
\textbf{Objetivo}: Buscar un objeto en múltiples fuentes de datos como archivos de los Satélites Spitzer y Herchel. \\
\textbf{Actor}: Usuario.\\
\textbf{Necesidad}: Esencial.\\
\textbf{Prioridad}: Media.\\
\textbf{Requerimientos Referenciados}: 7. \\
\textbf{Descripción}: El usuario ingresa al Portal Web, rellena los campos de nombre o código de objeto o coordenadas junto con un radio y se realiza la búsqueda. El sistema deberá realizar la búsqueda en base a ello en múltiples fuentes de información. Si se ingresa el nombre o código, se realiza una búsqueda teniendo en cuenta el margen error de cada fuente de datos.
\vspace{1.0cm}

\subsection{Simulaciones}
\noindent\textbf{Caso de uso \#21}: \\
\textbf{Objetivo}: Realiza una búsqueda de una simulación. \\
\textbf{Actor}: Usuario.\\
\textbf{Necesidad}: Deseable.\\
\textbf{Prioridad}: Baja.\\
\textbf{Requerimientos Referenciados}: 8. \\
\textbf{Descripción}: El usuario ingresa al Portal Web y realiza un búsqueda de una simulación.
\vspace{1.0cm}

\subsection{Servicios Bibliográficos}
\noindent\textbf{Caso de uso \#22}: \\
\textbf{Objetivo}: Rl realizar una búsqueda, desplegar en los resultados enlaces a información bibliográfica de SIMBAD. \\
\textbf{Actor}: Usuario.\\
\textbf{Necesidad}: Deseable.\\
\textbf{Prioridad}: Media.\\
\textbf{Requerimientos Referenciados}: 9. \\
\textbf{Descripción}: El usuario ingresa al Portal Web y al desplegarse los resultados de una búsqueda, cada resultado contiene un enlace a una búsqueda con todos los datos contenidos en SIMBAD, ADS y/o NED pertenecientes al/los objeto/s.
